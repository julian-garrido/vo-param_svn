\documentclass[a4paper,11pt] {article}
\usepackage{geometry}                % See geometry.pdf to learn the layout options. There are lots.
\geometry{letterpaper}                   % ... or a4paper or a5paper or ... 
%\geometry{landscape}                % Activate for for rotated page geometry
%\usepackage[parfill]{parskip}    % Activate to begin paragraphs with an empty line rather than an indent
\usepackage{graphicx}
\usepackage{amssymb}
\usepackage{epstopdf}
\usepackage{amsmath}
\usepackage{hyperref}
\usepackage{color}

\DeclareGraphicsRule{.tif}{png}{.png}{`convert #1 `dirname #1`/`basename #1 .tif`.png}

\title{The Parameter Description Language. }
\author{Carlo Maria Zw\"olf, Paul Harrison and Franck Le Petit}
%\date{}                                           % Activate to display a given date or no date

\begin{document}
\maketitle

\tableofcontents

\section{Introduction}
Bla Bla on PDL

\section{Basic concepts and general ideas}


\section{The SingleParameter Object}\label{par01}
The {\it SingleParameter} object is the core element for describing jobs. 
Every object of this type must be characterized by:
\begin{itemize}
\item A name (which is unique and is the Id of the parameter);
\item A unique parameter type, which explain the nature of the current parameter. The admitted types are : boolean, string, rational, complex, integer, real, date;
\item A unique dimension. A $1$-dimension corresponds to a scalar parameter whereas a dimension equal to N corresponds to a N-size vector. The dimension is expressed using an {\it expression} (cf. paragraph \ref{par02}). The result of the expression that appears in this {\it SingleParameter}-field object {\bf must be integer}.\footnote{This is obvious, since this value corresponds to a vector size.} 
\end{itemize}
The unique attribute {\it dependency} can take one of the two values {\bf required} or {\bf optional}. If required the parameter {\bf must be} provided to the service. If optional, the service could work even without the current parameter and the values will be considered for processing only if provided.\\

Optional fields for the {\it SingleParameter} object are:
\begin{itemize}
\item a unique UCD : which is a reference to an existing UCD for characterizing the parameter (\textcolor{red}{to be extended});
\item a unique Utype  : which is a reference to an existing Utype for characterizing the parameter (\textcolor{red}{to be extended});
\item a unique SkossConcempt (\textcolor{red}{to be extended}).
\item a unique Unit (\textcolor{red}{to be extended}).
\item a unique precision. This field must be specified only for parameter types where precision-concepts has a sense. It has indeed no sense for integer, rational or string. It has sense, for instance, on real type. For understanding the meaning of this field, let the function $f$ be a model of a given service. If $i$ denotes the input parameter, $f(i)$ denotes the output. The precision $\delta$ is the smaller value such that $f(i+\delta) \neq f(i)$.\\ The precision is expressed using an {\it expression} (cf. paragraph \ref{par02}). The result of the expression that appears in this {\it precision}-field  {\bf must be} of the same type than (i.e. could be naturally casted to) the type appearing in the field {\it parameter type}.
\end{itemize}

\section{The ParameterGroup object}\label{par-group}
The {\it ParameterGroup} object is used for grouping parameters according to a criterion of relevancy arbitrary chosen by users (for instance parameters may be grouped according to the physics : position-group, speed-group; thermodynamic-group).
But  the ParameterGroup is not only a kind of parameter set. Indeed it could be used for defining complex relations and/or constraints involving the contained parameters (cf. paraghrap \ref{par-ConstraintsOnGroup}).\\
This object {\bf must contain} a unique Name. This name is a String and is the identification label of the ParameterGroup. Indeed two groups can not have the same Name.\\
Optional fields are
\begin{itemize}
\item the references to the parameters (cf. paragraph XYZ) one want to include into the group;
\item a unique object ConstraintOnGroup of type {\it ConstraintOnGroup} (cf. paragraph  \ref{par-ConstraintsOnGroup}). This object is used for 
expressing the complex relations and constraints involving parameters.
\item the objects of type {\it ParametersGroup} contained into the current root group. Indeed the {\it ParametersGroup} is a recursive object which can contain other sub-groups.
\end{itemize}

{\bf NB:} A given {\it SingleParameter} object could only belong to one {\it ParameterGroup}. The reasons behind this rule will be more clear in paragraph XYZ.\\

{\bf NB:}  For any practical use, the number on the parameter referenced into a given group summed to the number of sub-groups of the same group must be greater than one. Otherwise the group would be a hollow shell.

\section{Expressing complex relations and constraints on parameters} 

\subsection{The ConstraintOnGroup Object}\label{par-ConstraintsOnGroup}
The {\it  ConstraintOnGroup} object is always contained into a {\it ParameterGroup} object and could not exist alone. 
This object {\bf must contain} the {\it ConditionalStatement} objects. These last are used, As it is shown in paragraph \ref{par-ConditionalStatement}, for expressing the complex relations and constraints involving parameters.

\subsection{The ConditionalStatement object}\label{par-ConditionalStatement}
The {\it ConditionalStatement} object, as its name indicates, are used for defining conditional statement. This object is abstract. In this section we are going to review the two concrete objects extending and specializing  {\it ConditionalStatement}. 

\subsubsection{The AlwaysConditionalStatement}\label{par-AlwaysConditionalStatement}
As its name indicates, this object is used for expressing statement that must always be valid. It {\bf must contain} a unique {\it Always} object (this last extends {\it ConditionalClause}, cf. paragraph \ref{par-ConditionalClause}).

\subsubsection{The IfThenConditionalStatement}\label{par-IfThenConditionalStatement}
As its name indicates, this object is used for expressing statements that are valid only if a previous condition is verified. It {\bf must contain}:
\begin{itemize}
\item a unique {\it If} object (this last extends {\it ConditionalClause}, cf. paragraph \ref{par-ConditionalClause}).
\item a unique {\it Then} object (this last extends {\it ConditionalClause}, cf. paragraph \ref{par-ConditionalClause}).
\end{itemize}
If the condition contained into the {\it If} object is valid, the condition contained into the {\it Then} object {\bf must be} valid too.

\subsection{The ConditionalClause object}\label{par-ConditionalClause}
The {\it ConditionalClause} object is abstract. It {\bf must contain} a unique Criterium object of type {\it AbstractCriterion} (cf. paragraph \ref{par-AbstractCriterion}).\\
The three concrete objects extending the abstract {\it ConditionalClause} are:
\begin{itemize}
\item {\it Always};
\item {\it If};
\item {\it Then}.
\end{itemize}
The Criterium contained into a {\it Always} object must always be valid (cf paragraph \label{par-AlwaysConditionalStatement}).\\
The {\it If} and {\it Then} objects work as a couple by composing the  {\it IfThenConditionalStatement} (cf. paragraph 
\ref{par-IfThenConditionalStatement}).

\subsection{The AbstractCriterion object}\label{par-AbstractCriterion}
The objects extending {\it AbstractCriterion} are fundamentals for building {\it ConditionalStatemets} (cf. paragraph \ref{par-ConditionalStatement}) since they are contained into the {\it Always, If} and {\it Then} objects (cf. paragraph \ref{par-ConditionalClause}).
An {\it AbstractCriterion} object {\bf must contain}:
\begin{itemize}
\item a unique {\it Expression} object (cf. paragraph \ref{par02});
\item a unique ConditionType which is an object of type {\it AbstractCondition} (cf. paragraph  \ref{par-ConditionType}).
This object specify which condition must be satisfies by the previous {\it Expression}.
\end{itemize}
An optional field is a unique {\it LogicalConnector} object (cf. paragraph \ref{par-LogicalConnector}) used for building logical expressions.\\
The two concrete objects extending  {\it AbstractCriterion}  are {\it Criterion} and {\it ParenthesisCriterion}.  The difference between these two objects is in the priority they induce for interpreting and linking the criterions (cf. paragraph \ref{par-EvalCriterions}).

\subsubsection{The Criterion object}
This object extends the  {\it AbstractCriterion} without specializing it. It is indeed just a concrete version of the abstract type.

\subsubsection{The ParenthesisCriterion object}
This object extends and specialize the  {\it AbstractCriterion}. It is used for defining arbitrary priority in interpreting boolean expression based on criterions.
The optional field of {\it ParenthesisCriterion} is a unique {\it ExternalLogicalConnector} object of type {\it LogicalConnector}. It is used for linking other criterions, out of the priority perimeter defined by the parenthesis (cf.  paragraph \ref{par-EvalCriterions}).

\subsection{The LogicalConnector object}\label{par-LogicalConnector}
The {\it LogicalConnector} object is used for building complex logical expressions. It is an abstract object and it {\bf must} contain a unique Criterion of type {\it AbstractCriterion} (cf. paragraph \ref{par-AbstractCriterion}).\\
The two concrete objects extending {\it LogicalConnector} are:
\begin{itemize}
\item the {\it And} object used for introducing the logical AND operator between two criterions;\footnote{The first criterion is the one containing the {\it LogicalConnector} and the second is the criterion contained into the connector itself.}
\item the {\it Or} object used for introducing the logical OR operator between two criterions.
\end{itemize}

\subsection{The AbstractCondition object}\label{par-ConditionType}
{\it AbstractCondition} is abstract type. The objects extending it always always belong to an  {\it  AbstractCriterion} (cf. \ref{par-AbstractCriterion}). In this context, they are used combined with an {\it Expression} object, for expressing the condition that the expression must satisfy.\\ 
Let us consider a given criterion object $\mathcal{CR}$ (extending{\it AbstractCriterion})  and let us note $\mathcal E$ and
$\mathcal C$ the expression and the condition contained into $\mathcal{CR}$.
In what follows we are going to explain the different objects specializing  {\it AbstractCondition} and their behavior.

\subsubsection{The IsNull condition}\label{par-IsNull}
This object is used for specifying that the expression $\mathcal E$ has no assigned value (this is exactly the same concept as the NULL value in Java or the None value in Python). 
Indeed, if and only if $\mathcal E$ has no assigned value, the evaluation of the couple $(\mathcal E, \mathcal C)$ leads to a TRUE boolean value. Thus, in the case $\mathcal{CR}$ has no {\it LogicalConnector}, the criterion is true.



\subsubsection{The "numerical-type" conditions}
These objects are used for specifying that the result of the evaluation of the expression $\mathcal E$ is of a given numerical type. The couple $(\mathcal E, \mathcal C)$ is legal if and only if $\mathcal E$ is a {\bf numerical} expression. \\
The "numerical-type" objects extending {\it AbstractCondition}  are:
\begin{itemize}
\item {\it IsInteger}, in this case the evaluation of the couple $(\mathcal E, \mathcal C)$ leads to a TRUE boolean value
if and only if the evaluation of the numerical expression $\mathcal E$ is an integer.
\item {\it IsRational}, in this case the evaluation of the couple $(\mathcal E, \mathcal C)$ leads to a TRUE boolean value if and only if the evaluation of the numerical expression $\mathcal E$ is a rational number.
\item {\it IsReal}, in this case the evaluation of the couple $(\mathcal E, \mathcal C)$ leads to a TRUE boolean value
if and only if the evaluation of the numerical expression $\mathcal E$ is real number.
\end{itemize} 

\subsubsection{The BelongToSet condition}\label{par-BelongToSet}
This object is used for specifying that the expression $\mathcal E$ could take only a finite set of values. 
It {\bf must contain} the {\it Values} (which are objects of type {\it Expression})  
defining the set of legal values. The number of {\it Values} must be greater than one.\\
This object is legal only if all the {\it Expressions} of the set are of the same type (e.g. they are all numerical, or all boolean or all String expressions).\\
The couple $(\mathcal E, \mathcal C)$ leads to a TRUE boolean value if and only if 
\begin{itemize}
\item the expression $\mathcal E$ and the expressions composing the set are of the same type.
\item it exists an element $\mathcal E_s$ in the set such that $\mathcal E_s = \mathcal E$.\\
This last equality is to be unserstood in 
the following sense: let $=_t$ be the equality operator induced by the type (for numerical type the equality is in the real number sense, for String type the equality is case sensitive and for boolean the equality is in the classic boolean sense).\\
Two expressions are equals if and only if
\begin{itemize}
\item the expressions has the same size $d_{\mathcal E}$,
\item $\mathcal E_s^i =_t \mathcal E^i$, $\forall i =1,...,d_{\mathcal E}$, where $\mathcal E_s^i$ and $ \mathcal E^i$ are respectively the result of the evaluation of the 
$i$ component of expressions $\mathcal E_s$ and $\mathcal E$. 
\end{itemize}
\end{itemize}

\subsubsection{The ValueLargerThan object}
This object is used for expressing that the result of the evaluation of the expression $\mathcal E$ 
must be greater than a given value.\\
It {\bf must contain}
\begin{itemize}
\item a unique {\bf numerical} expression $\mathcal E_c$. 
\item a unique {\it Reached} attribute which is a boolean type.
\end{itemize}
The couple $(\mathcal E, \mathcal C)$ is legal only if $\mathcal E$ is a numerical expression.\\
This couple leads to a TRUE boolean value if and only if 
the result of the evaluation of the expression $\mathcal E$ is greater (resp. greater or equal) than the result of the evaluation of the expression $\mathcal E_c$ and the attribute {\it Reached} is false (resp. the {\it Reached} attribute is true).

\subsubsection{The ValueSmallerThan object}
This object is used for expressing that the result of the evaluation of the expression $\mathcal E$ 
must be smaller than a given value.\\
It {\bf must contain}
\begin{itemize}
\item a unique {\bf numerical} expression $\mathcal E_c$. 
\item a unique {\it Reached} attribute which is a boolean type.
\end{itemize}
The couple $(\mathcal E, \mathcal C)$ is legal only if $\mathcal E$ is a numerical expression.\\
This couple leads to a TRUE boolean value if and only if 
the result of the evaluation of the expression $\mathcal E$ is smaller (resp. smaller or equal) than the result of the evaluation of the expression $\mathcal E_c$ and the attribute {\it Reached} is false (resp. the {\it Reached} attribute is true).

\subsubsection{The ValueInRange object}
This object is used for expressing that the result of the evaluation of the expression $\mathcal E$ must belong to a given interval. The definition of the interval is made using the {\it ValueLargerThan} {\it ValueSmallerThan} objects. 
Indeed, the {\it ValueInRange} object {\bf must contain}:
\begin{itemize}
\item a unique {\it ValueLargerThan} object,
\item a unique {\it ValueSmallerThan} object. 
\end{itemize}
The couple $(\mathcal E, \mathcal C)$ is legal only if $\mathcal E$ is a numerical expression.\\
This couple leads to a TRUE boolean value if and only if the evaluation of both couple 
$(\mathcal E, \mbox{\it ValueSmallerThan})$ and $(\mathcal E, \mbox{\it ValueLargerThan})$ lead to TRUE boolean values.

\subsubsection{The ValueDifferentOf object}
This object is used for specifying that the expression $\mathcal E$ must be different of a given value.
It {\bf must contain} a unique {\it Expression} $\mathcal E_c$.\\
In order to be compared, the two expressions $\mathcal E$ and $\mathcal E_c$ must have the same type.
The evaluation of the couple $(\mathcal E, \mathcal C)$  leads to a TRUE boolean value only if $\mathcal E \neq \mathcal E_c$. This inequality has to be understood in the sense explained in paragraph \ref{par-BelongToSet} (at the second point of the itemize).

\subsubsection{The DefaultValue object}\label{par-DefaultValue}
This object is used for specifying the default value of a parameter.\\
It {\bf must contain} a unique expression $\mathcal E_c$.\\
Since the default value of an expression involving functions, multiple parameters, etc. has no particular sense, in the case of the present object the couple $(\mathcal E, \mathcal C)$  is legal only if 
\begin{itemize}
\item $\mathcal E$ is an {\it AtomicParameterExpression} (cf. paragraph. \ref{par02_01})
\item the dimension and the type of the expression $\mathcal E_c$ are equal to the dimension and type expressed in the {\it SingleParameter} object referenced into the {\it AtomicParameterExpression}.
\end{itemize}
Moreover, for having a legal {\it DefaultValue} object, the criterion $\mathcal{CR}$ containing it must be contained into 
the {\it Always} or {\it Then}  objects (cf. paragraph \ref{par-ConditionalClause}).

\subsection{Evaluating and interpreting criterions objects}\label{par-EvalCriterions}
The evaluation of the criterion type objects (cf. paragraph \ref{par-AbstractCriterion}) always leads to a boolean value
(the only exception is what we saw in paragraph \ref{par-DefaultValue}, where the criterion contains a {\it DefaultValue} condition).\\
We use hereafter the same notation introduced in \ref{par-ConditionType}: let us consider a given criterion (extending{\it AbstractCriterion}) $\mathcal{CR}$ and let us note $\mathcal E$ and
$\mathcal C$ the expression and the condition contained into $\mathcal{CR}$.\\
When $\mathcal{CR}$ contains no {\it LogicalConnector} objects, the evaluation of the criterion is straightforward : 
the result is equal to the boolean-evaluation of the couple $(\mathcal E, \mathcal C)$. This couple is evaluated according to the concrete class involved, as explained in paragraphs \ref{par-IsNull} to \ref{par-DefaultValue} \\
It is a bit more complex when criterions contains {\it LogicalConnectors}. Let us see how to proceed.\\
To begin with, in a first time let us consider only {\it Criterion} concrete objects:\\
As we saw in the previous paragraphs, criterions object are (with the help of {\it LogicalConnectors} object) recursive and hierarchical objects.\\
This hierarchical structure composing a complex criterion could be graphically represented as follows.
\begin{equation}\label{eq-CriterionStructure01}
(\mathcal E_1, \mathcal C_1) \xrightarrow[{\mbox{\tiny AND/OR}}]{LC_1} 
(\mathcal E_2, \mathcal C_2) \xrightarrow[{\mbox{\tiny AND/OR}}]{LC_2}
\cdots 
(\mathcal E_i, \mathcal C_i) \xrightarrow[{\mbox{\tiny AND/OR}}]{LC_i}
\cdots 
(\mathcal E_{N-1}, \mathcal C_{N-1}) \xrightarrow[{\mbox{\tiny AND/OR}}]{LC_{N-1}}
(\mathcal E_N, \mathcal C_N)
\end{equation}
where the index $1$, $i$ and $N$ are respectively for the root, the $i$ and the leaf criterion composing the structure. The term $LC_i$ denotes the {\it LogicalConnector} contained into the criterion $\mathcal{CR}_i$.\\
As we saw in paragraphs \ref{par-IsNull} to \ref{par-DefaultValue} every couple $(\mathcal E_i, \mathcal C_i)$, $i=1,..,N$ could be evaluated (according to the concrete object involved) and leads to a boolean value $\mathcal B_i$. Thus the expression (\ref{eq-CriterionStructure01}) become
\begin{equation}
\mathcal B_1 \xrightarrow[{\mbox{\tiny AND/OR}}]{LC_1} 
\mathcal B_2 \xrightarrow[{\mbox{\tiny AND/OR}}]{LC_2}
\cdots 
\mathcal B_i \xrightarrow[{\mbox{\tiny AND/OR}}]{LC_i}
\cdots 
\mathcal B_{N-1} \xrightarrow[{\mbox{\tiny AND/OR}}]{LC_{N-1}}
\mathcal B_N
\end{equation}
This last is a classic sequential boolean expression. It is evaluated from left to right and the operator AND takes precedence over the OR operator.\\

Let us now consider complex criterions involving not only {\it Criterion} objects but also {\it ParenthesisContent} ones. In this case a graphical linear representation of the complex-criterion structure is the following:
\begin{equation}
\begin{array}{l}
\ (\mathcal E_1, \mathcal C_1) \xrightarrow{LC_1} 
(\mathcal E_2, \mathcal C_2) \xrightarrow{LC_2}
\cdots 
\\
\hspace{2cm}(\mathcal E_{i-1}, \mathcal C_{i-1}) \xrightarrow{LC_{i-1}}
(\mathcal E_i, \mathcal C_i) \xrightarrow{LC_i}
(\mathcal E_{i+1}, \mathcal C_{i+1}) \xrightarrow{LC_{i+1}}\\
 \hspace{4cm} \cdots (\mathcal E_{N-1}, \mathcal C_{N-1}) \xrightarrow{LC_{N-1}}
(\mathcal E_N, \mathcal C_N)\\
\end{array}
\end{equation}




\section{The Expression Objects}\label{par02}
The {\it Expression} is the more versatile component of the PDL. It occurs almost everywhere: in defining fields for {\it SingleParameters} (cf. paragraph \ref{par01}) or in defining conditions and criteria).\\
Expression is an abstract object. In this section we are going to review all the concrete object extending and specializing expression.\\

\noindent {\bf N.B.} In what follows, we will call a {\bf numerical expression} every {\it expression} involving only numerical types. This means that the evaluation of such expressions should lead to a number (or a vector number if the dimension of the expression is greater than one).\\


\subsection{The atomic parameter expression}\label{par02_01}
The {\it AtomicParameterExpression} (extending {\it Expression}) is the simplest expression that could be built involving a defined parameter. This object {\bf must contains} unique reference to a given parameter.\\

Optional fields, valid only for numerical types, are : 
\begin{itemize}
\item A unique {\bf numerical} power expression;
\item A unique operation (cf. paragraph \ref{par02_02}).\\
\end{itemize}
Let $p$ and $exp$ be respectively the parameter and the power expression we want to encapsulate. The composite object could be presented as follows:
\begin{equation}\label{eq01}
 \underbrace{  p^{exp} \underbrace{  \overbrace{\left( \begin{array}{c} + \\ - \\ \ast  \\ \cdot \\ \div   \end{array} \right) }^{\mbox{\tiny Operation type}}
 \overbrace{    \left( \mbox{AnotherExpression}\right) }^{\mbox{\tiny expression contained in operation}}   }_{\mbox{\tiny Operation object}}}_{\mbox{\tiny Atomic Parameter Expression}}
\end{equation}

To evaluate a given {\it atomicParameterExpression}, one proceed as follows: 
Let $d_p$, $d_{exp}$ and $d_{oo}$ be respectively the dimension of the parameter $p$ referenced, the dimension of the power expression and the dimension of the expression contained into the operation object.\\
The exponent part of the expression is legal if and only if:
\begin{itemize}
\item $d_p=d_{exp}$. In this case $p^{exp}$ is a $d_p$-size vector expression and $\forall$ $i=1,...,d_p$ the $i$ component of this vector is equal to ${p_i}^{exp_i}$, where $p_i$ is the value of the $i$ component of vector parameter $p$ and $exp_i$ is the value obtained by interpreting the $i$ component of vector expression $exp$.
\item $d_{exp}=1$. In this case, $\forall$ $i=1,...,d_p$ the $i$ component of the vector result is equal to ${p_i}^{exp}$, where $p_i$ is the same as defined above.\\
\end{itemize} 

Whatever the method is used, let us note $ep$ the result of this first step. Is is clear that the dimension of $ep$ 
is always equal to $d_p$. In order to complete the evaluation of the expression, one should proceed as exposed in paragraph \ref{par02_02}, by taking there $b=ep$.

\subsection{The atomic constant expression}\label{par02_03}
The {\it AtomicConstantExpression} (extending {\it Expression}) is the simplest expression that could be built involving constants. Since this object could be used for defining constant vector expression, it {\bf must contain} 
\begin{itemize}
\item A unique list of String which express the value of each component of the expression. Let $d_c$ be the size of the String list. If $d_c=1$ the expression is scalar and it is a vector expression if $d_c>1$. 
\item A unique attribute {\it constantType} of type {\it ParameterType} (cf. paragraph XYZ) which explain the nature of the constant expression. Then the admitted types are the same as in the field {\it parameterType} of the object {\it SingleParameter}. 
\end{itemize}
The object {\bf is legal if and only if} every element of the String list could be cast into the type expressed by the attribute {\it constantType}.\\

Optional fields, valid only for numerical types, are : 
\begin{itemize}
\item A unique {\bf numerical} power expression;
\item A unique operation (cf. paragraph \ref{par02_02}).
\end{itemize}

Let $s_i$ ($i=1,...,d_c$) and $exp$ be respectively the $i$ component of the String list and the power expression we want to encapsulate. The composite object could be presented as follows:
\begin{equation}
 \underbrace{    \overbrace{ \left( s_1  , s_2, ..., s_{d_c}\right)^{exp} }^{\mbox{\tiny List of String  to cast into the provided type}}   \underbrace{  \overbrace{\left( \begin{array}{c} + \\ - \\ \ast  \\ \cdot \\ \div   \end{array} \right) }^{\mbox{\tiny Operation type}}
 \overbrace{    \left( \mbox{AnotherExpression}\right) }^{\mbox{\tiny expression contained in operation}}   }_{\mbox{\tiny Operation object}}}_{\mbox{\tiny Atomic Constant Expression}}
\end{equation}
To evaluate a given {\it atomicConstantExpression}, one proceed as follows: let $d_{exp}$ and $d_{oo}$ be respectively the dimension of the parameter $p$ referenced, the dimension of the power expression and the dimension of the expression contained into the operation object.\\
The exponent part of the expression is legal if and only if:
\begin{itemize}
\item $d_c = d_{exp}$. In this case $(s_1,...,s_{d_c})^{exp}$ is a $d_c$ size vector expression and $\forall i =1,...,d_c$ the $i$-th component of this vector is equal to $s_i^{exp_i}$, where $exp_i$ is the value obtained by interpreting the $i$ component of vector $exp$.
\item $d_{exp}=1$. In this case, $\forall i =1,...,d_c$ the $i$ component of the vector result is equal to $s_i^{exp}$.\\
\end{itemize}
Whatever the method is used, let us note $ep$ (whose dimension is always equal to $d_c$) the result of 
this first step.  In order to complete the evaluation of the expression, one should proceed as exposed in paragraph \ref{par02_02}, by taking there $b=ep$.


\subsection{The parenthesis content expression}\label{par02_04}
The {\it parenthesisContentExpression} (extending {\it Expression}) object is used to explicitly denote precedence by grouping the expressions that should be evaluated first. This object {\bf must contain} a unique {\bf numerical} object $Expression$ (noted hereafter $exp_1$).\\
Optional fields are 
\begin{itemize}
\item A unique {\bf numerical} power expression (noted hereafter $exp_2$);
\item A unique operation (cf. paragraph \ref{par02_02}).\\
\end{itemize}
This composite object could be presented as follows:
\begin{equation}
 \underbrace{    \underbrace{ \left( exp_1\right) }_{\mbox{\tiny Priority term}} ^{\hspace{15mm}exp_2} \underbrace{  \overbrace{\left( \begin{array}{c} + \\ - \\ \ast  \\ \cdot \\ \div   \end{array} \right) }^{\mbox{\tiny Operation type}}
 \overbrace{    \left( \mbox{AnotherExpression}\right) }^{\mbox{\tiny expression contained in operation}}   }_{\mbox{\tiny Operation object}}}_{\mbox{\tiny Parenthesis Expression}}
\end{equation}
In order to evaluate this object expression, one proceed as follows: first one evaluate the expression $exp_1$ that has the main priority. Then one proceed exactly as in paragraph \ref{par02_01} (after the equation (\ref{eq01})) by taking $p=exp_1$ and $exp=exp_2$.


%Suppose we've already defined the ContainerExp  and the Contained_Expression.

\subsection{The Operation object}\label{par02_02}
The {\it Operation} object is used for expressing operations involving two {\bf numerical} expressions. This object {\bf must contain}:
\begin{itemize}
\item a unique attribute operationType. This attribute could take the following values: plus for the sum, minus for the difference, multiply for the standard product, scalarProduct for the scalar product and divide for the standard division. Hereafter these operators will be respectively denoted $+,-,\ast,\cdot, \div$. 
\item a unique expression.
\end{itemize}
\begin{equation}\label{OperationEquation}
\underbrace{ \overbrace{\left( \begin{array}{c} + \\ - \\ \ast  \\ \cdot \\ \div   \end{array} \right) }^{\mbox{\tiny Operation type}}
 \overbrace{    \left( \mbox{ContaindedExpression}\right) }^{\mbox{\tiny expression contained in operation}}   }_{\mbox{\tiny Operation object}}
\end{equation}

The {\it Operation} object is always contained into a {\bf numerical} {\it Expression} (cf. paragraph \ref{par02}) and could not exists alone.
Let $a$ be the result of the evaluation of the expression object containing the operation\footnote{this came from the evaluation of parameterRef field in case of an {\it AtomicParameterExpression} cf. paragraph \ref{par02}, from the evaluation of constant field in the case of a {\it AtomicConstantExpression} (\textcolor{red}{to be extended...})} and let $b$ the result of the evaluation of the {\bf numerical} expression contained into the operation. As usual, we note $d_a$ and $d_b$ the dimensions of $a$ and $b$.\\

The operation evaluation is legal if and only if:
\begin{itemize}
\item $d_a=d_b$ and operation type (i.e. the operator) $op \in \{ + , - , \ast , \div \}$. In this case $a \, op \, b$ is a vector expression of size $d_a$ and $\forall$ $i=1,...,d_a$ the $i$ component of this vector is equal to
$(a_i \, op \,b_i)$ (i.e. a term by term operation).
\item $d_a = d_b$ and operation type $op$ is "$\cdot$". In this case $a \cdot b$ is the result of the scalar product $\sum_{i =1}^{d_a} a_i \ast b_i$. It is obvious that the dimension of this result is equal to $1$.
\item $d_b=1$ and operation type (i.e. the operator) $op \in \{ + , - , \ast , \div \}$. In this case $a \, op \, b$ is a vector expression of size $d_a$ and $\forall$ $i=1,...,d_a$ the $i$ component of this vector is equal to
$(a_i \, op \,b)$.
\item $d_a=1$ and operation type (i.e. the operator) $op \in \{ + , - , \ast , \div \}$. This case in symmetric to the previous one.
\end{itemize}

The type of the result is automatically induced by standard cast operation performed during the evaluations (Indeed for example a double vector added to an integer vector is a double vector).  

\subsection{The Function object}\label{par02_05}
The {\it function} object (extending {\it expression}) is used for expressing mathematical function on expressions.
This object {\bf must contain} 
\begin{itemize}
\item A unique attribute {\it functionName} of type {\it functionType} (cf. paragraph XYZ) which specify the nature of the function.
\item A unique expression (which is the function argument).\\
\end{itemize}
Let $exp$ be the result of the evaluation of the function argument expression and note $d_{exp}$ its dimension.
The {\it function} object evaluation {\bf is legal if and only if}:
\begin{itemize}
\item $f  \in \{ \mbox{abs, sin, cos, tan, asin, acos, atan, exp, log} \}$ and the function argument is a {\bf numerical} expression. In this case the  result  is a $d_{exp}$-size vector and each component  $r_i = f(exp_i$), $\forall \, i=1,...,d_{exp}$.
\item $f=$sum (resp. $f=$product) and the argument is a {\bf numerical} expression. In this case the result is a scalar value equal to $\sum_{i=1}^{i=d_{exp}} exp_i$ (resp. $\prod _{i=1}^{i=d_{exp}} exp_i$), where $exp_i$ is the value obtained 
by interpreting the $i$ component of vector expression $exp$.
\item $f=$size. In this case the result is the scalar integer value $d_{exp}$. 
\end{itemize}
From what we saw above, the result of the interpretation of a function object { \bf is always a number}.

\subsection{The FunctionExpression object} 
The {\it FunctionExpression} object (extending {\it Expression}) is used for building mathematical expressions involving functions.\\
This object {\bf must contains} a unique {\it Function} object (cf. paragraph \ref{par02_05}).\\
Optional fields, valid only for numerical types, are : 
\begin{itemize}
\item A unique {\bf numerical} power expression;
\item A unique operation (cf. paragraph \ref{par02_02}).\\
\end{itemize}

This composite object could be presented as follows:
\begin{equation}
 \underbrace{    \underbrace{ \left(\mbox{function}\right) }_{\mbox{\tiny Function object}} ^{\hspace{15mm}exp} \underbrace{  \overbrace{\left( \begin{array}{c} + \\ - \\ \ast  \\ \cdot \\ \div   \end{array} \right) }^{\mbox{\tiny Operation type}}
 \overbrace{    \left( \mbox{AnotherExpression}\right) }^{\mbox{\tiny expression contained in operation}}   }_{\mbox{\tiny Operation object}}}_{\mbox{\tiny FunctionExpression Object}}
\end{equation}
In order to evaluate this object expression, one proceed as follows: first one evaluate the funtion expression as explained in paragraph  \ref{par02_05}. Then one proceed exactly as in paragraph \ref{par02_01} (after the equation (\ref{eq01})) by taking $p=$function.

\end{document}



 
